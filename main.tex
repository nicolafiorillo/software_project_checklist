% Load the kaobook class
\documentclass[
	fontsize=10pt, % Base font size
	twoside=false, % Use different layouts for even and odd pages (in particular, if twoside=true, the margin column will be always on the outside)
	%open=any, % If twoside=true, uncomment this to force new chapters to start on any page, not only on right (odd) pages
	secnumdepth=1, % How deep to number headings. Defaults to 1 (sections)
]{kaobook}

% Choose the language
\usepackage[english]{babel} % Load characters and hyphenation
\usepackage[english=british]{csquotes}	% English quotes

% Load packages for testing
\usepackage{blindtext}
%\usepackage{showframe} % Uncomment to show boxes around the text area, margin, header and footer
%\usepackage{showlabels} % Uncomment to output the content of \label commands to the document where they are used

% Load the bibliography package
\usepackage{kaobiblio}
\addbibresource{minimal.bib} % Bibliography file

% Load mathematical packages for theorems and related environments
\usepackage{kaotheorems}

% Load the package for hyperreferences
\usepackage{kaorefs}

\graphicspath{{images/}{./}} % Paths where images are looked for

\makeindex[columns=3, title=Alphabetical Index, intoc] % Make LaTeX produce the files required to compile the index


\begin{document}

%----------------------------------------------------------------------------------------
%	BOOK INFORMATION
%----------------------------------------------------------------------------------------

\titlehead{v. 0.1-draft}
\title[Software Project Checklist]{Software Project Checklist}
\author[JBG]{Nicola Fiorillo}
\date{\today}
\publishers{}

%----------------------------------------------------------------------------------------

\frontmatter % Denotes the start of the pre-document content, uses roman numerals

%----------------------------------------------------------------------------------------
%	COPYRIGHT PAGE
%----------------------------------------------------------------------------------------

\makeatletter
\uppertitleback{\@titlehead} % Header

\lowertitleback{	
	\textbf{No copyright} \\
	\cczero\ This book is released into the public domain using the CC0 code. To the extent possible under law, I waive all copyright and related or neighbouring rights to this work.
	
	To view a copy of the CC0 code, visit: \\\url{http://creativecommons.org/publicdomain/zero/1.0/}
	
	\medskip
	
	\textbf{Colophon} \\
	This document was typeset with the help of \href{https://sourceforge.net/projects/koma-script/}{\KOMAScript} and \href{https://www.latex-project.org/}{\LaTeX} using the \href{https://github.com/fmarotta/kaobook/}{kaobook} class.
}
\makeatother

%----------------------------------------------------------------------------------------
%	DEDICATION
%----------------------------------------------------------------------------------------

% \dedication{
% 	The harmony of the world is made manifest in Form and Number, and the heart and soul and all the poetry of Natural Philosophy are embodied in the concept of mathematical beauty.\\
% 	\flushright -- D'Arcy Wentworth Thompson
% }

%----------------------------------------------------------------------------------------
%	OUTPUT TITLE PAGE AND PREVIOUS
%----------------------------------------------------------------------------------------

% Note that \maketitle outputs the pages before here
\maketitle

%----------------------------------------------------------------------------------------
%	PREFACE
%----------------------------------------------------------------------------------------

% \chapter*{Preface}

% Goal is to have a tool to verify every single aspect of a (ideally software) project.
% This document is a continuous work-in-progress.

%----------------------------------------------------------------------------------------
%	TABLE OF CONTENTS & LIST OF FIGURES/TABLES
%----------------------------------------------------------------------------------------

\begingroup % Local scope for the following commands

	% Define the style for the TOC, LOF, and LOT
	%\setstretch{1} % Uncomment to modify line spacing in the ToC
	%\hypersetup{linkcolor=blue} % Uncomment to set the colour of links in the ToC
	\setlength{\textheight}{230\vscale} % Manually adjust the height of the ToC pages

	% Turn on compatibility mode for the etoc package
	\etocstandarddisplaystyle % "toc display" as if etoc was not loaded
	\etocstandardlines % "toc lines as if etoc was not loaded

	\tableofcontents % Output the table of contents

	% \listoffigures % Output the list of figures

	% Comment both of the following lines to have the LOF and the LOT on different pages
	% \let\cleardoublepage\bigskip
	% \let\clearpage\bigskip

	% \listoftables % Output the list of tables

\endgroup

%----------------------------------------------------------------------------------------
%	MAIN BODY
%----------------------------------------------------------------------------------------

\mainmatter % Denotes the start of the main document content, resets page numbering and uses arabic numbers
\setchapterstyle{kao} % Choose the default chapter heading style

\chapter{Introduction and goals}
Goal is to have a tool to verify every single aspect of a (ideally software) project.

Currently this manufact is a work-in-progress.

\chapter{Business}

	\section{Why(s)}
		\begin{itemize}
			\item[-] Why building \textbf{this} solution?
			\item[-] Why \textbf{building} this solution?
		\end{itemize}

		\section{Other}
		\begin{itemize}
			\item[-] Defining solution lifetime
			\item[-] Cost analisys
		\end{itemize}

\chapter{Solution goals}

\section{User goals}
	\begin{itemize}
		\item[-] Defining personas
		\item[-] Analyze user goals
		\item[-] Analyze user environment
		\begin{itemize}
			\item Define involved areas
			\item Define communication channels
		\end{itemize}
	\end{itemize}

\section{Product goals}
	\begin{itemize}
		\item[-] Define commitments
		\item[-] Define scope(s)
		\item[-] Define a clear project vision and mission
		\item[-] Define (high level) functional requirements
		\item[-] Define (high level) non-functional requirements
		\item[-] Requirements not to be considered
	\end{itemize}

\chapter{People}
	\begin{itemize}
		\item[-] Project responsible
		\item[-] Decision-capable (and reachable) contact persons (for budget-related decisions)
		\item[-] Domain experts (and reachable)
		\item[-] Product owner
		\item[-] Developers
		\item[-] Project leader
		\item[-] Testers
		\item[-] Designers
		\item[-] Devops
		\item[-] Support (level 1 and 2)
		\item[-] Other experts
		\item[-] Defining future maintainers
	\end{itemize}

\chapter{Analysis}
	\begin{itemize}
		\item[-] Alternative solutions/competitors investigation
			\begin{itemize}
				\item Taking inspiration
				\item Pro vs. cons analysis
			\end{itemize}
		\item[-] Defining user stories
		\item[-] Lifecycle model and management
			\begin{itemize}
				\item Wartefall/iteration approach
				\item Issue tracking tool
				\item Sharing knowledge, sharing code ownership
				\item Backlog and issue prioritization (urgency vs. severity/impact)
			\end{itemize}
		\item[-] Defining solution roadmap
		\item[-] Defining project phases/milestones
			\begin{itemize}
				\item Verifiable (and undisputable) outcomes
				\item Prefer short phases
			\end{itemize}
		\item[-] Define a release strategy
		\item[-] Functional requirements
			\begin{itemize}
				\item System administration functionalities
			\end{itemize}

		\item[-] Non-functional requirements
			\begin{itemize}
				\item Defining product type (web, mobile, desktop, other)
				\item Deployments
					\begin{itemize}
						\item cloud
						\item on-premise
					\end{itemize}
				\item Security requirements
				\item Privacy requirements (GDPR)
					\begin{itemize}
						\item data protection
					\end{itemize}
				\item Legal requirements
				\item Logging requirements
				\item Auditing requirements
				\item Monitoring/alert (telemetry) requirements
					\begin{itemize}
						\item defining KPIs and metrics
						\item defining plan
					\end{itemize}
				\item Accessibility requirements
					\begin{itemize}
						\item responsiveness
					\end{itemize}
				\item Error handling (?)
				\item API semantic consensus
				\item Load estimation
					\begin{itemize}
						\item number of concurrent users (average, peaks)
						\item API requests per second (average, peaks)
						\item data quantity
						\item bandwidth (data transfer)
						\item cpu usage
					\end{itemize}
				\item Scalability strategies
				\item Recovery strategies (in case of failures)
				\item External dependencies involved
				\item Defining failure scenarios
				\item Declaring well-known limitations
			\end{itemize}
		\item[-] Define a deploy strategy
			\begin{itemize}
				\item installation requirements (env target)
				\item update/migration requirements (max downtime accepted)
				\item define rollback plan
				\item deploy documentation (release notes)
			\end{itemize}
		\item[-] Define a QA strategy
			\begin{itemize}
				\item accessibility, load, security, etc...
			\end{itemize}
		\item[-] Education/training requirements
		\item[-] Tech decisions
			\begin{itemize}
				\item operating systems, frameworks, architectural patterns, programming languages, data persistences
				\item external dependencies communication
			\end{itemize}
		\item[-] Risks analisys
			\begin{itemize}
				\item cost-benefit analysis
			\end{itemize}
		\item[-] Use decision records documentation: \href{https://github.com/joelparkerhenderson/architecture-decision-record}{architecture decision record}
		\item[-] UX analisys (iteratively with domain experts)
			\begin{itemize}
				\item goal: became a domain expert involving domain experts)
				\item defining personas, create workflow/mockups
				\item usability test with final users
				\item acceptance test (user stories) with final users
			\end{itemize}
		\item[-] Brainstorming/discussion/shared solution
			\begin{itemize}
				\item goal: sharing domain problems, improve common domain/product knowledge
			\end{itemize}
		\item[-] UI design
			\begin{itemize}
				\item brand manual, color guide
			\end{itemize}
	\end{itemize}

\chapter{Development}
	\begin{itemize}
		\item[-] Create alias for email-team (inside and outside communication)
		\item[-] Team and stakeholders shall be informed of:
			\begin{itemize}
				\item project vision and mission
				\item constraints
				\item deadlines
			\end{itemize}
		\item[-] Defining automated test strategies (functional, integration, unit, UI, load, performances, security, system)
		\item[-] Devops strategies
			\begin{itemize}
				\item defining source control repository strategy (monorepo)
				\item build/test/deploy automation tools
				\item defining environments (on-click enviroment deploy, creation/update)
					\begin{itemize}
						\item local
						\item development
						\item test
						\item staging
						\item production
					\end{itemize}
			\end{itemize}
		\item[-] Defining code review/approval strategy
		\item[-] Define the DoD (Definition of Done)
			\begin{itemize}
				\item workflows/mockups (UX)
				\item product artifacts
				\item documentations (User guide, API service, technical documents, legal conformity to)
				\item acceptance tests positive results
			\end{itemize}
		\item[-] Code formatting standards consensus
			\begin{itemize}
				\item linters
			\end{itemize}
		\item[-] API guideline consensus
		\item[-] API definition
			\begin{itemize}
				\item API signatures must came from use cases
				\item document API from the beginning
			\end{itemize}
		\item[-] Dev principles
			\begin{itemize}
				\item KISS principle
				\item Occam's razor
				\item Unit test
				\item git usage principles
				\item every commit shall contain only one modification (feature, bug fix, task) by transient branch
			\end{itemize}
		\item[-] Syncronization
			\begin{itemize}
				\item daily standup/sync meeting
				\item periodic demo to customer/management (iteration/sprint-related)
				\item retrospective
			\end{itemize}
	\end{itemize}

% \pagelayout{wide} % No margins
% \addpart{Title of the Part}
% \pagelayout{margin} % Restore margins

\chapter{Deploy and Release}
\blindtext

\chapter{Notes}
	\begin{itemize}
		\item[-] Technical specs document: \href{https://stackoverflow.blog/2020/04/06/a-practical-guide-to-writing-technical-specs/}{stackoverflow, a practical guide to writing technical specs}
	\end{itemize}

% \appendix % From here onwards, chapters are numbered with letters, as is the appendix convention

% \pagelayout{wide} % No margins
% \addpart{Appendix}
% \pagelayout{margin} % Restore margins

%----------------------------------------------------------------------------------------

\backmatter % Denotes the end of the main document content
\setchapterstyle{plain} % Output plain chapters from this point onwards

%----------------------------------------------------------------------------------------
%	BIBLIOGRAPHY
%----------------------------------------------------------------------------------------

% The bibliography needs to be compiled with biber using your LaTeX editor, or on the command line with 'biber main' from the template directory

\defbibnote{bibnote}{Here are the references in citation order.\par\bigskip} % Prepend this text to the bibliography
\printbibliography[heading=bibintoc, title=Bibliography, prenote=bibnote] % Add the bibliography heading to the ToC, set the title of the bibliography and output the bibliography note

%----------------------------------------------------------------------------------------
%	INDEX
%----------------------------------------------------------------------------------------

% The index needs to be compiled on the command line with 'makeindex main' from the template directory

\printindex % Output the index

\end{document}
